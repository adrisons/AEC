Radix sort es un algoritmo de ordenación cuyo rendimiento temporal en el peor caso es de $\mathcal{O}(kN)$ y de memoria $\mathcal{O}(k + N)$

Su principal característica es la utilización de un array de 10 posiciones denominado \emph{bucket}, inicializado en cada una de las iteraciones a 0 y que guarda en cada una de las posiciones del array el número de apariciones de cada una de las cifras coincidentes con la propia posición del \emph{bucket}.

Este \emph{bucket} es utilizado para calcular el propio \emph{bucket} acumulado con el que asignaremos una posición diferente a cada uno de los elementos en el array en \emph{b}.

En cada una de las iteraciones tenemos una variable llamada \emph{exp} que se irá multiplicando por 10 y que nos indica la cifra que miraremos en ese instante y que luego copiaremos a \emph{a}, que es el array inicial.

Este es el código de dicho algoritmo:

\begin{lstlisting}
// Uso del bucket en radix sort
while (m / exp > 0){
	int bucket[10] = {0};
	for (i = 0; i < n; i++){
		bucket[a[i] / exp % 10]++;
	}
	for (i = 1; i < 10; i++)
		bucket[i] += bucket[i - 1];
	for (i = n - 1; i >= 0; i--)
		b[--bucket[a[i] / exp % 10]] = a[i];
	for (i = 0; i < n; i++){
		a[i] = b[i];
	}
	exp *= 10;
}
\end{lstlisting}


\subsection{Ejemplo del algoritmo}

Vector inicial: 25 57 48 37 12 92 86 33\\

Los elementos quedarían ordenados de la siguiente manera:\\
0:\\
1:\\
2: 1\underline{2} 9\underline{2}\\
3: 3\underline{3}\\
4:\\
5: 2\underline{5}\\
6: 8\underline{6}\\
7: 5\underline{7} 3\underline{7}\\
8: 4\underline{8}\\
9:\\

El bucket, en vez de ser lo descrito anteriormente, quedaría con el número de elementos asignados a cada posición:\\
\\
0: 0\\
1: 0\\
2: 2\\
3: 1\\
4: 0\\
5: 1\\
6: 1\\
7: 2\\
8: 1\\
9: 0\\
\\

El vector después de esta iteración sería: 12 92 33 25 86 57 37 48\\

En la siguiente iteración nos centramos el la segunda cifra de cada uno de los elementos:\\
\\
0:\\
1: \underline{1}2\\
2: \underline{2}5\\
3: \underline{3}3 \underline{3}7\\
4: \underline{4}8\\
5: \underline{5}7\\
6: \\
7: \\
8: \underline{8}6\\
9: \underline{9}2\\

En esta ocasión el bucket quedaría de la siguiente forma:
\\
0: 0\\
1: 1\\
2: 1\\
3: 2\\
4: 1\\
5: 1\\
6: 0\\
7: 0\\
8: 1\\
9: 1\\

En este ejemplo el vector ya ha quedado ordenado: 12 25 33 37 48 57 86 92\\



\subsection{La complejidad del algoritmo}
